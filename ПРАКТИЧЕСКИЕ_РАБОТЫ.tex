\documentclass[12pt,a4paper,oneside]{article}
\usepackage[russian,shorthands=off]{babel}
\usepackage[utf8]{inputenc}
\usepackage[T1]{fontenc}
\usepackage{amsmath}
\usepackage{amssymb}
\usepackage{geometry}
\geometry{verbose,a4paper,tmargin=2.5cm,bmargin=2.5cm,lmargin=2.5cm,rmargin=2.5cm}
\usepackage{fancyhdr}
\usepackage{array}
\usepackage{colortbl}
\usepackage[dvipsnames]{xcolor}
\usepackage{booktabs}
\usepackage{multirow}
\usepackage{makecell}

\pagestyle{fancy}
\fancyhf{}
\fancyhead[L]{Вычислительная техника}
\fancyhead[C]{Практические работы}
\fancyhead[R]{\thepage}
\renewcommand{\headrulewidth}{0.5pt}

\title{\textbf{ПРАКТИЧЕСКИЕ РАБОТЫ ПО КУРСУ\\''ВЫЧИСЛИТЕЛЬНАЯ ТЕХНИКА''}}
\author{}
\date{21 января 2026 г.}

\begin{document}

\maketitle
\tableofcontents
\newpage

% ========== ПРАКТИЧЕСКАЯ РАБОТА № 1 ==========
\section{Практическая работа № 1: Анализ логических выражений и функций}

\subsection{Задание 1: Логическое выражение}

\subsubsection{Условие задачи}

Дано логическое выражение:
\[
\overline{a\overline{bc} + \overline{ab}c} + a\overline{\overline{b}}c = y
\]

\subsubsection{Решение}

Упростим выражение пошагово:

\begin{enumerate}
    \item Раскроем двойное отрицание: $a\overline{\overline{b}} = ab$
    
    \item Исходное выражение примет вид:
    \[
    \overline{a\overline{bc} + \overline{ab}c} + abc = y
    \]
    
    \item Применим закон де Моргана к части в скобках:
    \[
    \overline{a\overline{bc} + \overline{ab}c} = \overline{a\overline{bc}} \cdot \overline{\overline{ab}c}
    \]
    
    \item Получим:
    \[
    = (\overline{a} + b + c) \cdot (a + \overline{b} + \overline{c})
    \]
    
    \item Полное выражение:
    \[
    y = (\overline{a} + b + c)(a + \overline{b} + \overline{c}) + abc
    \]
\end{enumerate}

\subsection{Задание 2: Анализ функции F}

\subsubsection{Исходная функция}

\[
F = \overline{\left(\overline{A} \land B\right) \lor \left(B \land C\right)} \lor \left(\overline{A} \land C\right)
\]

\subsubsection{Пошаговое упрощение}

Обозначим промежуточные функции:
\begin{align*}
X &= \overline{A} \land B \\
Y &= B \land C \\
Z &= X \lor Y \\
U &= \overline{Z} \\
V &= \overline{A} \land C \\
F &= U \lor V
\end{align*}

\subsubsection{Таблица истинности функции F}

\begin{center}
\begin{tabular}{|c|c|c|c|c|c|c|c|c|c|}
\hline
$A$ & $B$ & $C$ & $\overline{A}$ & $X$ & $Y$ & $Z$ & $U$ & $V$ & $F$ \\
\hline
0 & 0 & 0 & 1 & 0 & 0 & 0 & 1 & 0 & \textbf{1} \\
\hline
0 & 0 & 1 & 1 & 0 & 0 & 0 & 1 & 1 & \textbf{1} \\
\hline
0 & 1 & 0 & 1 & 1 & 0 & 1 & 0 & 0 & \textbf{0} \\
\hline
0 & 1 & 1 & 1 & 1 & 1 & 1 & 0 & 1 & \textbf{1} \\
\hline
1 & 0 & 0 & 0 & 0 & 0 & 0 & 1 & 0 & \textbf{1} \\
\hline
1 & 0 & 1 & 0 & 0 & 0 & 0 & 1 & 0 & \textbf{1} \\
\hline
1 & 1 & 0 & 0 & 0 & 0 & 0 & 1 & 0 & \textbf{1} \\
\hline
1 & 1 & 1 & 0 & 0 & 1 & 1 & 0 & 0 & \textbf{0} \\
\hline
\end{tabular}
\end{center}

\subsubsection{Минимальная дизъюнктивная нормальная форма (МДНФ)}

Из таблицы видно, что функция F = 1 для наборов:
\begin{itemize}
    \item (0,0,0): $\overline{A}\overline{B}\overline{C}$
    \item (0,0,1): $\overline{A}\overline{B}C$
    \item (0,1,1): $\overline{A}BC$
    \item (1,0,0): $A\overline{B}\overline{C}$
    \item (1,0,1): $A\overline{B}C$
    \item (1,1,0): $AB\overline{C}$
\end{itemize}

\[
F = \overline{A}\overline{B}(\overline{C} + C) + \overline{A}BC + A\overline{B}(C + \overline{C}) + AB\overline{C}
\]

\[
F = \overline{A}(\overline{B} + BC) + A(\overline{B} + B\overline{C})
\]

\[
F = \overline{A}(\overline{B} + C) + A(\overline{B} + \overline{C})
\]

% ========== ПРАКТИЧЕСКАЯ РАБОТА № 2 ==========
\newpage
\section{Практическая работа № 2: Системы счисления}

\subsection{Перевод чисел в восьмеричную систему счисления}

\subsubsection{Задача 1: Перевод числа $2112{,}77_{10}$ в восьмеричную систему}

\paragraph{Шаг 1: Перевод целой части (деление на 8)}

\begin{center}
\begin{tabular}{|c|c|c|c|}
\hline
\textbf{Делимое} & \textbf{÷ 8} & \textbf{Частное} & \textbf{Остаток} \\
\hline
2112 & & 264 & \textbf{0} \\
\hline
264 & & 33 & \textbf{0} \\
\hline
33 & & 4 & \textbf{1} \\
\hline
4 & & 0 & \textbf{4} \\
\hline
\end{tabular}
\end{center}

\textbf{Результат целой части:} остатки снизу вверх = $4100_8$

\paragraph{Шаг 2: Перевод дробной части (умножение на 8)}

\begin{center}
\begin{tabular}{|c|c|c|c|}
\hline
\textbf{Дробь $\times$ 8} & \textbf{Целая часть} & \textbf{Новая дробь} & \textbf{Цифра} \\
\hline
$0{,}77 \times 8 = 6{,}16$ & 6 & 0,16 & \textbf{6} \\
\hline
$0{,}16 \times 8 = 1{,}28$ & 1 & 0,28 & \textbf{1} \\
\hline
$0{,}28 \times 8 = 2{,}24$ & 2 & 0,24 & \textbf{2} \\
\hline
$0{,}24 \times 8 = 1{,}92$ & 1 & 0,92 & \textbf{1} \\
\hline
$0{,}92 \times 8 = 7{,}36$ & 7 & 0,36 & \textbf{7} \\
\hline
\end{tabular}
\end{center}

\textbf{Результат дробной части:} $0{,}61217_8$ (первые 5 цифр)

\paragraph{Ответ:}
\[
2112{,}77_{10} = 4100{,}61217_8
\]

\subsubsection{Задача 2: Перевод числа $6422{,}64_{10}$ в восьмеричную систему}

\paragraph{Шаг 1: Перевод целой части (деление на 8)}

\begin{center}
\begin{tabular}{|c|c|c|c|}
\hline
\textbf{Делимое} & \textbf{÷ 8} & \textbf{Частное} & \textbf{Остаток} \\
\hline
6422 & & 802 & \textbf{6} \\
\hline
802 & & 100 & \textbf{2} \\
\hline
100 & & 12 & \textbf{4} \\
\hline
12 & & 1 & \textbf{4} \\
\hline
1 & & 0 & \textbf{1} \\
\hline
\end{tabular}
\end{center}

\textbf{Результат целой части:} остатки снизу вверх = $14426_8$

\paragraph{Шаг 2: Перевод дробной части (умножение на 8)}

\begin{center}
\begin{tabular}{|c|c|c|c|}
\hline
\textbf{Дробь $\times$ 8} & \textbf{Целая часть} & \textbf{Новая дробь} & \textbf{Цифра} \\
\hline
$0{,}64 \times 8 = 5{,}12$ & 5 & 0,12 & \textbf{5} \\
\hline
$0{,}12 \times 8 = 0{,}96$ & 0 & 0,96 & \textbf{0} \\
\hline
$0{,}96 \times 8 = 7{,}68$ & 7 & 0,68 & \textbf{7} \\
\hline
$0{,}68 \times 8 = 5{,}44$ & 5 & 0,44 & \textbf{5} \\
\hline
$0{,}44 \times 8 = 3{,}52$ & 3 & 0,52 & \textbf{3} \\
\hline
\end{tabular}
\end{center}

\textbf{Результат дробной части:} $0{,}50753_8$

\paragraph{Ответ:}
\[
6422{,}64_{10} = 14426{,}50753_8
\]

% ========== ПРАКТИЧЕСКАЯ РАБОТА № 3 ==========
\newpage
\section{Практическая работа № 3: Цифровые логические схемы}

\subsection{Логические вентили}

\subsubsection{Вентиль И (AND)}

\textbf{Функция:} $Y = A \land B$

\textbf{Таблица истинности:}

\begin{center}
\begin{tabular}{|c|c|c|}
\hline
$A$ & $B$ & $Y$ \\
\hline
0 & 0 & 0 \\
\hline
0 & 1 & 0 \\
\hline
1 & 0 & 0 \\
\hline
1 & 1 & 1 \\
\hline
\end{tabular}
\end{center}

\subsubsection{Вентиль ИЛИ (OR)}

\textbf{Функция:} $Y = A \lor B$

\textbf{Таблица истинности:}

\begin{center}
\begin{tabular}{|c|c|c|}
\hline
$A$ & $B$ & $Y$ \\
\hline
0 & 0 & 0 \\
\hline
0 & 1 & 1 \\
\hline
1 & 0 & 1 \\
\hline
1 & 1 & 1 \\
\hline
\end{tabular}
\end{center}

\subsubsection{Вентиль НЕ (NOT)}

\textbf{Функция:} $Y = \overline{A}$

\textbf{Таблица истинности:}

\begin{center}
\begin{tabular}{|c|c|}
\hline
$A$ & $Y$ \\
\hline
0 & 1 \\
\hline
1 & 0 \\
\hline
\end{tabular}
\end{center}

\subsubsection{Вентиль ИСКЛЮЧАЮЩЕЕ ИЛИ (XOR)}

\textbf{Функция:} $Y = A \oplus B = A\overline{B} + \overline{A}B$

\textbf{Таблица истинности:}

\begin{center}
\begin{tabular}{|c|c|c|}
\hline
$A$ & $B$ & $Y$ \\
\hline
0 & 0 & 0 \\
\hline
0 & 1 & 1 \\
\hline
1 & 0 & 1 \\
\hline
1 & 1 & 0 \\
\hline
\end{tabular}
\end{center}

\subsection{Триггеры}

\subsubsection{RS-триггер (Reset-Set)}

\textbf{Описание:} Базовый триггер с двумя входами (Set и Reset) и двумя выходами (Q и $\overline{Q}$).

\textbf{Таблица переходов:}

\begin{center}
\begin{tabular}{|c|c|c|c|}
\hline
$S$ & $R$ & $Q_{t+1}$ & \textbf{Состояние} \\
\hline
0 & 0 & $Q_t$ & Сохранение \\
\hline
0 & 1 & 0 & Сброс \\
\hline
1 & 0 & 1 & Установка \\
\hline
1 & 1 & ? & Запрещено \\
\hline
\end{tabular}
\end{center}

% ========== ПРАКТИЧЕСКАЯ РАБОТА № 4 ==========
\newpage
\section{Практическая работа № 4: Сумматоры и арифметические схемы}

\subsection{Полусумматор (Half Adder)}

\textbf{Функция:} Суммирует два одноразрядных двоичных числа без учета переноса на вход.

\textbf{Таблица истинности:}

\begin{center}
\begin{tabular}{|c|c|c|c|}
\hline
$A$ & $B$ & $S$ (Sum) & $C$ (Carry) \\
\hline
0 & 0 & 0 & 0 \\
\hline
0 & 1 & 1 & 0 \\
\hline
1 & 0 & 1 & 0 \\
\hline
1 & 1 & 0 & 1 \\
\hline
\end{tabular}
\end{center}

\textbf{Логические выражения:}
\begin{align*}
S &= A \oplus B \quad \text{(сумма)} \\
C &= A \land B \quad \text{(перенос)}
\end{align*}

\subsection{Полный сумматор (Full Adder)}

\textbf{Функция:} Суммирует два одноразрядных двоичных числа с учетом переноса на вход.

\textbf{Таблица истинности:}

\begin{center}
\begin{tabular}{|c|c|c|c|c|}
\hline
$C_{in}$ & $A$ & $B$ & $S$ (Sum) & $C_{out}$ (Carry) \\
\hline
0 & 0 & 0 & 0 & 0 \\
\hline
0 & 0 & 1 & 1 & 0 \\
\hline
0 & 1 & 0 & 1 & 0 \\
\hline
0 & 1 & 1 & 0 & 1 \\
\hline
1 & 0 & 0 & 1 & 0 \\
\hline
1 & 0 & 1 & 0 & 1 \\
\hline
1 & 1 & 0 & 0 & 1 \\
\hline
1 & 1 & 1 & 1 & 1 \\
\hline
\end{tabular}
\end{center}

\textbf{Логические выражения:}
\begin{align*}
S &= (A \oplus B) \oplus C_{in} \quad \text{(сумма)} \\
C_{out} &= AB + (A \oplus B)C_{in} \quad \text{(перенос)}
\end{align*}

% ========== ПРАКТИЧЕСКАЯ РАБОТА № 5 ==========
\newpage
\section{Практическая работа № 5: Встроенные цифровые схемы}

\subsection{Дешифратор (Decoder)}

\textbf{Назначение:} Преобразует n-битный код на входе в один активный выход из $2^n$ возможных выходов.

\textbf{2-в-4 дешифратор:}

\begin{center}
\begin{tabular}{|c|c|c|c|c|c|}
\hline
$A_1$ & $A_0$ & $Y_0$ & $Y_1$ & $Y_2$ & $Y_3$ \\
\hline
0 & 0 & 1 & 0 & 0 & 0 \\
\hline
0 & 1 & 0 & 1 & 0 & 0 \\
\hline
1 & 0 & 0 & 0 & 1 & 0 \\
\hline
1 & 1 & 0 & 0 & 0 & 1 \\
\hline
\end{tabular}
\end{center}

\subsection{Мультиплексор (Multiplexer)}

\textbf{Назначение:} Выбирает один из нескольких входов на основе адреса выбора и направляет его на выход.

\textbf{4-в-1 мультиплексор:}

\begin{center}
\begin{tabular}{|c|c|c|}
\hline
$S_1$ & $S_0$ & $Y$ \\
\hline
0 & 0 & $I_0$ \\
\hline
0 & 1 & $I_1$ \\
\hline
1 & 0 & $I_2$ \\
\hline
1 & 1 & $I_3$ \\
\hline
\end{tabular}
\end{center}

\textbf{Логическое выражение:}
\[
Y = \overline{S_1}\overline{S_0}I_0 + \overline{S_1}S_0I_1 + S_1\overline{S_0}I_2 + S_1S_0I_3
\]

\subsection{Компаратор (Comparator)}

\textbf{Назначение:} Сравнивает два входных слова и выдает результат сравнения (A>B, A=B, A<B).

\textbf{1-битный компаратор:}

\begin{center}
\begin{tabular}{|c|c|c|c|c|}
\hline
$A$ & $B$ & $A>B$ & $A=B$ & $A<B$ \\
\hline
0 & 0 & 0 & 1 & 0 \\
\hline
0 & 1 & 0 & 0 & 1 \\
\hline
1 & 0 & 1 & 0 & 0 \\
\hline
1 & 1 & 0 & 1 & 0 \\
\hline
\end{tabular}
\end{center}

% ========== РЕЗЮМЕ ==========
\newpage
\section{Резюме}

Данный документ содержит полное описание основных практических работ по курсу ``Вычислительная техника'':

\begin{description}
    \item[Практическая работа № 1] посвящена анализу логических выражений и булевых функций с применением законов де Моргана и построением таблиц истинности.
    
    \item[Практическая работа № 2] охватывает перевод чисел между позиционными системами счисления (десятичная, восьмеричная) для целых и дробных частей.
    
    \item[Практическая работа № 3] описывает базовые логические вентили (И, ИЛИ, НЕ, XOR), триггеры (RS-триггер) и их применение в цифровых схемах.
    
    \item[Практическая работа № 4] содержит информацию о сумматорах (полусумматор и полный сумматор) и их использовании в арифметических схемах.
    
    \item[Практическая работа № 5] посвящена встроенным цифровым схемам: дешифраторам, мультиплексорам и компаратором.
\end{description}

\vspace{2cm}

\begin{center}
    \rule{6cm}{1pt} \\
    \textit{Документ составлен:} 21 января 2026 г. \\
    \textit{Предмет:} Вычислительная техника \\
    \textit{Курс:} Основы цифровых схем и логики
\end{center}

\end{document}
